\chapter{COMMODIFIED DIGITAL MARKETPLACE THEORY}
\section{Precursor}

The theory for this model is a compilation of my personal experience with digital marketplaces and research done under Jason Abaluck at the Yale School of Management. For the past 6 years, I have built and refined a digital marketplace for student labor. Local students build their work profiles and community members connect with them for all sorts of jobs. Hire a Student (\url{https://hireastudent.org}) has over 12,000 registered users worldwide and growing. Hire is somewhere between a diversified and commodified marketplace, as community members do connect directly with the student, but the price is set by the marketplace for the given job. Marketplaces are always much more complex than these reductionist classifications, but they provide a narrow framework to think about the economic incentives at play.

\section{Model}

\subsection{Setup and Assumptions}

This is a simple model of a Commodified Digital Marketplace, such as Wag or Lugg. As stated above, a Commodified Digital Marketplace sets the price of the service indepent of each supplier. Here are the assumptions that underpin the CDM Theory:

\begin{itemize}
  \item The market for this service already exists
  \item There is sufficient supply in the market
  \item Marketplace is aggregating supply for a specific service and setting the price
\end{itemize}

\subsection{Definitions}

To set the stage, we'll walk through the variables used in this theory

\subsubsection{Service}

The service exists by endogenous assumption in this model, and we are assuming that there is product market fit. Examples of commodified services are included in Figure \ref{digital_marketplaces} 

\subsubsection{Suppliers}

Suppliers are individuals or businesses that provide the service. They are approved and managed by the marketplace. Each supplier has a production capacity. All suppliers that fail to provide a high quality service do not get added to the supply force. Supply is generally the most important part of any marketplace, and marketplaces must at once acquire new suppliers while maintaining quality

\[g \textrm{: Supplier production capacity}\]
\[S' \textrm{: New suppliers added in a given period}\]
\[S_n \textrm{: Existing suppliers in period n}\]
\[f \textrm{: Failure rate of new suppliers}\]
\[C_S \textrm{: Cost of supplier acquisition}\]
\[C_n \textrm{: Total marketplace cost in period n}\]
\[Q_n \textrm{: Quantity of service available in period n}\]

\subsubsection{Quality}

Quality is commodified and is evaluated for each service provided by the marketplace.

\[X_H \textrm{: High quality}\]
\[X_L \textrm{: Low quality}\]
\[X_M \textrm{: Marketplace quality as a function of } S'\]

\subsubsection{Price}

Price is calculated proportional to the expected quality of the suppliers. A consumer expects that Uber drivers will be somewhere between a limosine and a tired taxicab.
This model assumes a very simple linear pricing model proportional to the quality provided.

\[P_H \textrm{: High quality price}\]
\[P_L \textrm{: Low quality price}\]
\[P_n \textrm{: Marketplace price as a function of } S' \textrm{ in period n}\]

This is clearly a simplification, as most marketplaces are very thoughtful in picking their price in order to optimize their revenue (see model extensions).

\subsubsection{Demand}

By assumption, there already exists a market for this service, so consumers are willing to buy from the marketplace as long as the price is appropriately scaled.

\subsubsection{Marketplace}

The marketplace organizes the provision and payment of this service, paying to acquire suppliers and earning transaction fees. The marketplace earns and maximizes profit.

\[z \textrm{: Marketplace transaction fee}\]
\[\Pi_n \textrm{: Profit in period n}\]

\subsection{Equations}

The marketplace price is a function of the expected quality, which is calculated linearly

\[P_n(S') = \dfrac{S'*(P_H*(1-f) + P_L*f) + S_n * P_H}{S' + S_n} \]

\vspace{5 mm}
The quantity of the service is calculated from the supply production capacity and numbers of suppliers

\[Q_n(S') = (S_n + S') * g \]

\vspace{5 mm}
The only cost in this model is to acquire new suppliers

\[C_n(S') = C_S * S' \]

\vspace{5 mm}
Profit is revenue from transaction fees on the marketplace minus costs

\[\Pi_n(S') = P_n(S') * Q_n(S') * z - C(S') \]


\subsection{Solving the model}

Substituting in the formulas for price, quantity, and cost

\[ \Pi_n(S') = \dfrac{S'*(P_H*(1-f) + P_L*f) + S_n * P_H}{S' + S_n} * (S_n + S') * g * z - C_S * S' \]

\vspace{5 mm}
Reduces down to

\[ \Pi_n(S') = (S'*(P_H*(1-f) + P_L*f) + S_n * P_H ) * g * z - C_S * S' \]

\vspace{5 mm}
Deriving with respect to new suppliers

\[ \dfrac{\partial{\Pi_n}}{\partial{S'}} = P_H * (1-f) * g * z + P_L * f * g * z - C_S \]

\vspace{5 mm}
We set this partial derivative equal to zero in order to find the point where incremental profitability flips

\[ \dfrac{\partial{\Pi_n}}{\partial{S'}} = P_H * (1-f) * g * z + P_L * f * g * z - C_S = 0 \]

\[ P_H * (1-f) * g * z + P_L * f * g * z = C_S \]

\vspace{5 mm}
When incremental profitability is positive, the marketplace can grow profitably. When incremental profitability is negative, the marketplace cannot grow profitably in this period, but will likely be able to grow profitably across multiple periods.

\[ P_H * (1-f) * g * z + P_L * f * g * z > C_S \implies \textrm{short-run profitable growth} \]

\[ P_H * (1-f) * g * z + P_L * f * g * z < C_S \implies \textrm{no short-run profitable growth} \]

If there is no short-run profitable growth, the viability of the digital marketplace depends on the lifetime value of each customer and competition.

\subsection{Trade-offs}

There are a few notable trade-offs that this model presents:

\subsubsection{Capacity of marketplace suppliers vs. expected quality and price of service}

As the marketplace adds new suppliers, it dilutes its supply pool and lowers expected quality, because these new suppliers are untested.

\subsubsection{Short-run profitability vs long-run profitability}

For some commodified digital marketplaces, market conditions $ P_L, P_H, f, g $ (as defined in the modeled variables) are such that adding suppliers is a profitable endeavor right from the beginning. In this case, growth is natural and the challenge becomes execution and growth acceleration through financing.

For others, acquiring new suppliers is not immediately profitable, as the upfront cost outweighs their short-run capacity to provide the service. In these cases, the case for growth must include borrowing, discounting, with a watchful eye for competition and attrition.

\section{Conclusions}

\subsection{Borrowing Constraint}
In commodified digital marketplaces, market conditions can necessitate borrowing in order to finance supplier acquisition in the short and long term. Perhaps the most notable takeaway from the results of this model: when you are building something that has lasting power (through network effects), it is effective to borrow massively in order to overspend and seed the market. This strategy is further explored in Blitzscaling by Reid Hoffman, where Reid recommends that companies in winner-take-all or winner-take-most markets prioritize speed over efficiency in order to capture a defensible market \citep{blitzscaling}.

\subsection{Runaway Markets}
One interesting takeaway from the CDM model is the potential for optimal market conditions where there is a path to short-run profitability. If the cost to acquire new suppliers is lower than the value of those suppliers' activity on the platform, it will be advantageous to scale quickly and unapologetically, deploying blitzscaling just as Reid prescribes \citep{blitzscaling}. This suggests some consistency with leading industry approaches.

\subsection{Gamblers' Markets}
When acquiring suppliers doesn't pay off in the short run, the Commodified Digital Marketplace becomes more challenging. This difference makes scale capital-intensive and risky. In Gamblers' Markets, lifetime value of a customer and the marketplace's retention become more vital. If competition enters and there are multiple marketplaces subsidizing their supplier acquisition, the marketplaces may sustain persistent losses. An example of this is Uber competing with other ride-sharing services across the world.

\subsection{Marketplace Stages}

\subsubsection{Before product-market fit}
When Commodified Digital Marketplaces are just getting started, this model will be grossly inadequate. Each marketplace has its own idiosynchrosies, and it is nearly impossible to evaluate a market accurately at a glance. Thus, many of the values necessary to use this model for market evaluation would be unkown in early stages.

\subsubsection{After product-market fit}
After the digital marketplace has some product-market fit, this model becomes more useful. With the product-market fit, the digital marketplace can start to measure important values such as supplier acquisition cost and supplier capacity. These values inform the incentives for the marketplace to grow and optimal choices for how many suppliers to add and what transaction fee to charge.

\subsubsection{Mature Stage}
As the digital marketplace matures, the market dynamics evolve and may make this model increasingly less relevant. As the number of suppliers increases, the effect on expected quality from adding new suppliers becomes more muted. To make this model more accurate for later stages, it would need to incorporate retention and competition.

\section{Model Extensions}

This model is highly simplistic and should be extended to fit a specific market. Here are a number of possible extensions:

\subsection{Non-linear supplier acquisition cost}
The cost to acquire suppliers is likely non-linear. Depending on the market, the cost could increase or decrease with scale.
If suppliers are few and far between, it may be more difficult to find each incremental supplier. For example:
\[ C_n(S') = C_S*S'^1.2 \]

If suppliers are abundant, it may become less costly as the marketplace's brand proliferates. For example:
\[ C_N(S') = C_S*S'^0.8 \]

\subsection{Variable quality}
Real-world quality is much more complex than this reductionist model. Most marketplaces have reviews and ratings that help the marketplace monitor and ensure quality services. Additionally, in some marketplaces, there may be variability in commodified prices, as consumers demand slightly different variants of a given service (eg. paying more for high quality yard work)

\subsection{Marketplace chooses price}

Instead of assuming that the price is a function of expected quality, it is probably more realistic to assume that the marketplace chooses a price in order to maximize its profits.

\subsubsection{Demand as a function of price}

With the marketplace choosing price, the model could be reformulated to have demand as a function of price. This would enable market clearing and a more practical approach to consumer interest. 

\subsubsection{Supply as a function of price}

Attracting suppliers could also be reformulated to depend on the price chosen by the marketplace. In this case, adding new suppliers also has market clearing dynamics, making the model more closely resemble the two-sided nature of these digital marketplaces.
